Рассмотрим написанную нами программу. Для начала требуется подключить 
необходимые заголовочные файлы:

\begin{verbatim}
    
  #include "stm32f0xx_ll_rcc.h"
  #include "stm32f0xx_ll_system.h"
  #include "stm32f0xx_ll_bus.h"
  #include "stm32f0xx_ll_gpio.h"
\end{verbatim}

Также для работы потребуются функции, организующие задержки. 
Наиболее простой вариант - реализовать простой холостой цикл. 
Главное отключить оптимизации компилятора, во избежание
выкидывания им этого цикла.
\begin{verbatim}
  void delay( void )
  {
    for (int i = 0; i < 600000; i++);
  }
  
  void delay10( void )
  {
    for (int i = 0; i < 6000; i++);
  }
\end{verbatim}


Напишем функцию, реализующую конфигурацию пинов $GPIO$:
\begin{verbatim}
  void gpio_config( void )
  {
    LL_AHB1_GRP1_EnableClock(LL_AHB1_GRP1_PERIPH_GPIOA);
    LL_AHB1_GRP1_EnableClock(LL_AHB1_GRP1_PERIPH_GPIOC);
    
    LL_GPIO_SetPinMode(GPIOC, LL_GPIO_PIN_9, LL_GPIO_MODE_OUTPUT);
    LL_GPIO_SetPinMode(GPIOC, LL_GPIO_PIN_8, LL_GPIO_MODE_OUTPUT);
    
    // GPIOB init
    LL_AHB1_GRP1_EnableClock(LL_AHB1_GRP1_PERIPH_GPIOB);
    LL_GPIO_SetPinMode(GPIOB, LL_GPIO_PIN_0, LL_GPIO_MODE_OUTPUT);
    LL_GPIO_SetPinMode(GPIOB, LL_GPIO_PIN_1, LL_GPIO_MODE_OUTPUT);
    LL_GPIO_SetPinMode(GPIOB, LL_GPIO_PIN_2, LL_GPIO_MODE_OUTPUT);
    LL_GPIO_SetPinMode(GPIOB, LL_GPIO_PIN_3, LL_GPIO_MODE_OUTPUT);
    LL_GPIO_SetPinMode(GPIOB, LL_GPIO_PIN_4, LL_GPIO_MODE_OUTPUT);
    LL_GPIO_SetPinMode(GPIOB, LL_GPIO_PIN_5, LL_GPIO_MODE_OUTPUT);
    LL_GPIO_SetPinMode(GPIOB, LL_GPIO_PIN_6, LL_GPIO_MODE_OUTPUT);
    LL_GPIO_SetPinMode(GPIOB, LL_GPIO_PIN_7, LL_GPIO_MODE_OUTPUT);
    LL_GPIO_SetPinMode(GPIOB, LL_GPIO_PIN_8, LL_GPIO_MODE_OUTPUT);
    LL_GPIO_SetPinMode(GPIOB, LL_GPIO_PIN_9, LL_GPIO_MODE_OUTPUT);
    LL_GPIO_SetPinMode(GPIOB, LL_GPIO_PIN_10, LL_GPIO_MODE_OUTPUT);
    LL_GPIO_SetPinMode(GPIOB, LL_GPIO_PIN_11, LL_GPIO_MODE_OUTPUT);
  }
\end{verbatim} 

Для отображения определённой цифры на семисегментном индикаторе, 
нам потребуется битовая маска. Для её вычисления 
напишем специальную функцию:
\begin{verbatim}
  uint16_t show_digit( uint32_t digit )
  {
    uint16_t s[] = {
    0x0400, // segment A - 0
    0x0040, // segment B - 1
    0x0008, // segment C - 2
    0x0002, // segment D - 3
    0x0001, // segment E - 4
    0x0200, // segment F - 5
    0x0010, // segment G - 6
    };
    
    switch (digit)
    {
    case 0:
      return s[0] | s[1] | s[2] | s[3] | s[4] | s[5];
    case 1:
      return s[1] | s[2];
    case 2:
      return s[0] | s[1] | s[6] | s[4] | s[3];
    case 3:
      return s[0] | s[1] | s[2] | s[3] | s[6];
    case 4:
      return s[1] | s[2] | s[5] | s[6];
    case 5:
      return s[0] | s[2] | s[3] | s[5] | s[6];
    case 6:
      return s[0] | s[2] | s[3] | s[4] | s[5] | s[6];
    case 7:
      return s[0] | s[1] | s[2];
    case 8:
      return s[0] | s[1] | s[2] | s[3] | s[4] | s[5] | s[6];
    case 9:
      return s[0] | s[1] | s[2] | s[3] | s[5] | s[6];
    default:
      return 0;
    }
  }
\end{verbatim}

Перейдём к написанию главной функции. Она будет осуществлять 
вывод четырёхзначного числа на экран семисегментного индикатора. 
Так как семисегметный индикатор может вывести только одно число за раз, 
то вывод четырёх цифр будет происходить путём 
многократного последовательного вывода всех цифр числа.
\begin{verbatim}  
  void show_number( uint32_t num )
  {
    uint16_t res[4] = 
    {
      [0] = show_digit(num % 10) | 0x0980,
      [1] = show_digit(num / 10 % 10) | 0x0920,
      [2] = show_digit(num / 100 % 10) | 0x08A0,
      [3] = show_digit(num / 1000 % 10) | 0x01A0,
    };
    
    static int i = 0;
    
    LL_GPIO_WriteOutputPort(GPIOB, res[i]);
    
    i = (i + 1) % 4;
  }
\end{verbatim}

Главная функция программы с реализацией антидребезга:
\begin{verbatim}
  int main( void )
  { 
    gpio_config();
    uint32_t counter = 0, is_pressed = 0, is_on = 0;
    uint32_t n = 0;
    
    while (1)
    {
      // 1 - on, 0 - off
      /* Check if button pressed */
      if (LL_GPIO_IsInputPinSet(GPIOA, LL_GPIO_PIN_0))
      {
        is_pressed = 1; 
        counter = 0; /* reset counter */
      }
     
      if (is_pressed)
      {
        counter++;
        delay10();
      }
      
      if (counter >= 5)
      {
        if (is_on)
        {
          LL_GPIO_ResetOutputPin(GPIOC, LL_GPIO_PIN_9);
          LL_GPIO_SetOutputPin(GPIOC, LL_GPIO_PIN_8);
        }
        else
        {
          LL_GPIO_SetOutputPin(GPIOC, LL_GPIO_PIN_9);
          LL_GPIO_ResetOutputPin(GPIOC, LL_GPIO_PIN_8);
        }
        // change state
        is_on = 1 - is_on;
      
        //LL_GPIO_WriteOutputPort(GPIOB, 0x0000);
        ++n;
        
        is_pressed = 0;
        counter = 0;
      }
      
      show_number(n);
    }
  }
  
  \end{verbatim}