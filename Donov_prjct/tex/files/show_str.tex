\begin{verbatim}
    
  Рассмотрим написанную нами программу. Для начала подключаем необходимые заголовочные файлы:
      
  #include "libs/stm32f0xx_ll_rcc.h"
  #include "libs/stm32f0xx_ll_system.h"
  #include "libs/stm32f0xx_ll_bus.h"
  #include "libs/stm32f0xx_ll_gpio.h"
  
  \end{verbatim}
  
  Также для работы потребуются функции, организующие задержки. Наиболее простой вариант - реализовать простой холостой цикл.
  
  \begin{verbatim}
      
  void delay()
  {
    for (int i = 0; i < 600000; ++i)
    {}
    return;
  }
  
  void delay_10ms()
  {
      for (int i = 0; i < 6000; ++i)
      {}
      return;
  }
  
  \end{verbatim}
  
  Заведем массив из констант, которые будут отвечать за конкретные пины. 
  \begin{verbatim}
      
  uint16_t sgmnts[7] =
  { 
    LL_GPIO_PIN_0, // E - segment 0
    LL_GPIO_PIN_1, // D - segment 1
    LL_GPIO_PIN_3, // C - segment 2
    LL_GPIO_PIN_4, // G - segment 3
    LL_GPIO_PIN_6, // B - segment 4
    LL_GPIO_PIN_9, // F - segment 5 
    LL_GPIO_PIN_10, // A - segment 6
  };
  
  \end{verbatim}
  
  Организуем функцию для вывода конкретного числа с использованием выше определенного массива $sgmnts$
  
  \begin{verbatim}
      
  uint16_t Show_digit(uint32_t digit)
  {
       switch (digit)
       {
       case 0: return (sgmnts[0] | sgmnts[1] | sgmnts[2] |
                       sgmnts[4] | sgmnts[6] | sgmnts[5]);
               
       case 1: return (sgmnts[4] | sgmnts[2]);
       
       case 2: return (sgmnts[6] | sgmnts[4] | sgmnts[3] |
                       sgmnts[0] | sgmnts[1]);
               
       case 3: return (sgmnts[4] | sgmnts[2] | sgmnts[6] | 
                       sgmnts[3] | sgmnts[1]);
                       
       
    
       case 4: return (sgmnts[4] | sgmnts[2] | sgmnts[5] | 
                       sgmnts[3]);
               
               
       case 5: return (sgmnts[6] | sgmnts[2] | sgmnts[5] | 
                       sgmnts[3] | sgmnts[1]);
               
       case 6: return (sgmnts[6] | sgmnts[5] | sgmnts[0] |
                       sgmnts[3] | sgmnts[2] | sgmnts[1]);
               
       case 7: return (sgmnts[4] | sgmnts[2] | sgmnts[6]);
                       
               
       case 8: return (sgmnts[1] | sgmnts[2] | sgmnts[3] |
                       sgmnts[4] | sgmnts[5] | sgmnts[6] |
                       sgmnts[0]);
         
       case 9: return (sgmnts[1] | sgmnts[2] | sgmnts[3] |
                       sgmnts[4] | sgmnts[5] | sgmnts[6]);
               
       default: return 0; //disable all segments 
            
       }
  }
  
  \end{verbatim}
  
  Далее идет функция для реализации динамического дисплея. Суть заключается в том, что мы выводим одно конкретное 4-х значное число на дислпей индикатора с помощью $Show\_digit(...)$, а затем в цикле осуществляем итерацию и переходим к следующему разряду/цифре.
  
  \begin{verbatim}
      
  void dyn_display(uint16_t num)
  {
    while(1)
    {
    
      uint16_t result[4] = 
      {
        [0] = Show_digit(num % 10) | 0x0980,                         
        [1] = Show_digit((num / 10) % 10) |  0x0920,         
        [2] = Show_digit((num / 100) % 10) |  0x08A0,     
        [3] = Show_digit((num / 1000) % 10) |  0x01A0,        
    
      };
      
      static int digit_num = 0;
    
      LL_GPIO_WriteOutputPort(GPIOB, result[digit_num]); 
      delay();
      
      digit_num = (digit_num + 1) % 4;
    }
      return;
  }
  
  \end{verbatim}
  
  Организуем функцию для вывода конкретной буквы с использованием выше определенного массива $sgmnts$
  
  \begin{verbatim}
      
  uint16_t Hello_people(uint8_t symbol)
  {
    switch(symbol)
    {
    case 'H': return sgmnts[5] | sgmnts[0] | sgmnts[3] 
                               | sgmnts[4] | sgmnts[2];
    
    case 'E': return sgmnts[6] |  sgmnts[3] | sgmnts[1] 
                               | sgmnts[0] | sgmnts[5];
    
    case 'L': return sgmnts[0] | sgmnts[1] | sgmnts[5];
    
    case 'O': return (sgmnts[0] | sgmnts[1] | sgmnts[2] |
                      sgmnts[4] | sgmnts[6] | sgmnts[5]);
              
    case ',': return LL_GPIO_PIN_2; 
    
    case 'P': return sgmnts[6] |  sgmnts[3] | sgmnts[4] 
                                | sgmnts[0] | sgmnts[5];
    
    case ' ': return 0x0000;
    }
  }
  
  \end{verbatim}
  
  Главная функция - вывод бегущей строки. По сути все то же самое, что и у $dyn\_display(...)$. Разве что стоит отметить чуть более сложную работу с массивом $char$'ов.
  
  \begin{verbatim}
      
  void Show_str( uint32_t cnt )
  {
    static int i = 0;
  
    uint8_t str[] = "HELLO PEOPLE ";
    
    uint16_t scr_pos[] = { 0x01A0, 0x08A0, 0x0920, 0x0980};
    uint16_t size = sizeof(str) / sizeof(char) - 1;
  
    uint8_t sym_pos = (cnt + i) % size;
   
    uint16_t pos_mask = scr_pos[i];
  
    LL_GPIO_WriteOutputPort(GPIOB, Hello_people(str[sym_pos]) | pos_mask);
    
    i = (i + 1) % 4;
  }
  
  \end{verbatim}
  
  Напишем функцию, реализующую подачу тактирование на модуль порта $GPIO$. Настроим непосредственно выводы со светодиодами в режим цифровой выход.
  
  \begin{verbatim}
      
  void gpio_config()
  {  
    LL_AHB1_GRP1_EnableClock(LL_AHB1_GRP1_PERIPH_GPIOA); 
    // enable tact port
    LL_AHB1_GRP1_EnableClock(LL_AHB1_GRP1_PERIPH_GPIOC);
    
    LL_GPIO_SetPinMode(GPIOC, LL_GPIO_PIN_8, LL_GPIO_MODE_OUTPUT);
    LL_GPIO_SetPinMode(GPIOC, LL_GPIO_PIN_0, LL_GPIO_MODE_INPUT); 
    //enable digital port
  
    
    LL_AHB1_GRP1_EnableClock(LL_AHB1_GRP1_PERIPH_GPIOB);
      
    LL_GPIO_SetPinMode(GPIOB, LL_GPIO_PIN_0, LL_GPIO_MODE_OUTPUT);
    LL_GPIO_SetPinMode(GPIOB, LL_GPIO_PIN_1, LL_GPIO_MODE_OUTPUT);
    LL_GPIO_SetPinMode(GPIOB, LL_GPIO_PIN_2, LL_GPIO_MODE_OUTPUT);
    LL_GPIO_SetPinMode(GPIOB, LL_GPIO_PIN_3, LL_GPIO_MODE_OUTPUT);
    LL_GPIO_SetPinMode(GPIOB, LL_GPIO_PIN_4, LL_GPIO_MODE_OUTPUT);
    LL_GPIO_SetPinMode(GPIOB, LL_GPIO_PIN_5, LL_GPIO_MODE_OUTPUT);
    LL_GPIO_SetPinMode(GPIOB, LL_GPIO_PIN_6, LL_GPIO_MODE_OUTPUT);
    LL_GPIO_SetPinMode(GPIOB, LL_GPIO_PIN_7, LL_GPIO_MODE_OUTPUT);
    LL_GPIO_SetPinMode(GPIOB, LL_GPIO_PIN_8, LL_GPIO_MODE_OUTPUT);
    LL_GPIO_SetPinMode(GPIOB, LL_GPIO_PIN_9, LL_GPIO_MODE_OUTPUT);
    LL_GPIO_SetPinMode(GPIOB, LL_GPIO_PIN_10, LL_GPIO_MODE_OUTPUT);
    LL_GPIO_SetPinMode(GPIOB, LL_GPIO_PIN_11, LL_GPIO_MODE_OUTPUT);
  
    return;
  }
  
  \end{verbatim}
  
  Вспомогательная функция для вывода бегущей строки.
  \begin{verbatim}
      
  void Show()
  {
    static uint32_t n = 0;
    
    for (int i = 0; i < 20000; ++i)
      Show_str(n);
    
    n++;
  }
  
  \end{verbatim}
  
  
  Главная функция программы с реализацией антидребезга:
  
  \begin{verbatim}
  
   int main()
  {
    uint32_t status = 0, on_ = 0;
    uint32_t counter = 0;
    uint32_t buf[1] = {0};
  
    
    gpio_config();
  
    
    while (1)
    {
      status = LL_GPIO_IsInputPinSet(GPIOA, LL_GPIO_PIN_0); 
      // check register IDR cond
   
      if (status)
      {
        counter++; 
        delay_10ms();
      }
          
      if (counter >= 5)
      {
        LL_GPIO_ResetOutputPin(GPIOC, LL_GPIO_PIN_8);
          
         while(1)
         Show();
       }
     
      else
        LL_GPIO_SetOutputPin(GPIOC, LL_GPIO_PIN_8);
    } 
    
    return 0;
  }
  
  
  \end{verbatim}
  