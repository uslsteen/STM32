\documentclass{article}
\usepackage[utf8]{inputenc}
\usepackage[14pt]{extsizes}
\usepackage[russian]{babel}
\usepackage{fullpage}
\usepackage{graphicx}
\graphicspath{{pictures/}}
\DeclareGraphicsExtensions{.pdf,.png,.jpg}
\usepackage[usenames]{color}
\usepackage{colortbl}
\usepackage[hidelinks]{hyperref,xcolor}
\usepackage{setspace,amsmath}
\usepackage{indentfirst}

\begin{document}
	\begin{titlepage}
			\begin{center}
				\large 	Московский физико-технический институт \\
				Физтех-школа радиотехники и компьютерных технологий \\
				\vspace{0.2cm}
				
				\vspace{4.5cm}
				\large (Микроконтроллеры) \\ \vspace{0.2cm}
				\LARGE \textbf{Morse-talk}
			\end{center}
			\vspace{2.3cm} \large
			
			\begin{center}
				Работу выполнили: \\
				Руденко Варвара, Б01-901\\
				Шакирзянов Искандер, Б01-902
				\vspace{10mm}		
				
			\end{center}
			
			\begin{center} \vspace{50mm}
				г. Долгопрудный \\
				2021 год
			\end{center}
		\end{titlepage}
		
\section{Цель:}
При помощи двух микроконтроллеров научиться принимать, передавать в зашифрованном виде, принимать и выводить на экран текстовые сообщения.
\section{Встретившиеся трудности:}
При работе с проектом встретили некоторые проблемы: для набора некоторого сообщения, человек должен обладать чувством ритма, ввиду взаимосвязи набираемой буквы от длительности удержания пользовательской кнопки.
\section{Схема установки:}
2 микроконтроллера STM32F051 на отладочной плате, oled-дисплей, макетные платы, провода
	\begin{figure}[h!]
		\center{\includegraphics[width=15cm]{pic/1}}
	\end{figure}
\newpage
\section{Отладочная плата STM32F0DISCOVERY:}
	\begin{itemize}
		\item Микроконтроллер STM32F051R8T6, до 48 МГц, 64 KB Flash, 8 KB RAM, корпус LQFP64
Встроенный эмулятор-отладчик ST-LINK/V2
		\item Питание платы осуществляется двумя способами: через USB и через внешний источник
питания
		\item Напряжение питание от внешнего источника 3 В и 5 В
		\item 4 светодиода: питание 3 В, USB соединение, вывод PC8 и вывод PC9
		\item 2 кнопки: Пользовательская и Reset
	\end{itemize}
	\begin{figure}[h!]
		\center{\includegraphics[width=15cm]{pic/2}}
	\end{figure}
\newpage
\section{Микроконтроллер STM32F051R8T6:}
STM32F051R8T6 - это микроконтроллер, включающий в себя высокопроизводительное 32-разрядное RISC-ядро ARM Cortex-M0, работающее на частоте 48 МГц, высокоскоростные встроенные запоминающие устройства и широкий спектр расширенных периферийных устройств и систем ввода-вывода. Все устройства предлагают стандартные коммуникационные интерфейсы: один 12-битный АЦП, один 12-битный ЦАП, шесть 16-битных таймеров, один 32-битный таймер и ШИМ-таймер с расширенным управлением.
	\begin{itemize}
		\item CRC calculation unit
		\item 4 to 32MHz Crystal oscillator
		\item 32kHz Oscillator for RTC with calibration
		\item Internal 8MHz RC with x6 PLL option
		\item Internal 40kHz RC oscillator
		\item 0 to 3.6V Conversion range
		\item 2.4 to 3.6V Separate analogue supply
		\item Independent and system watchdog timers
		\item SysTick timer
		\item Communication interfaces
		\item HDMI CEC, Wakeup on header reception
		\item Serial wire debug (SWD)
		\item 96-bit Unique ID
	\end{itemize}
	\begin{figure}[h!]
		\center{\includegraphics[width=15cm]{pic/3}}
	\end{figure}
\newpage
\section{Принцип работы:}
Transmitter – блок, принимающий от пользователя тектсовую информацию в виде сообщения зашифрованного азбукой Морзе, высвечивающий светодиодом считанное сообщение(для понимания, что мы отправляем), шифрующий с помощью кодового слова и передающий сообщение через USART. Для этого берется слово (в нашем примере это microcontroller), убираются повторяющиеся буквы и ставятся в начало алфавита, а после идут оставшиеся. В файле "decoder.txt" находится конфигурационные данные для шифровки/дешифровки. 
Reciever – блок, принимающий информацию через USART, дешифрующий ее и выводящий на oled-дисплей сообщение.
\section{Reciever:}
	\begin{figure}[h!]
		\center{\includegraphics[width=15cm]{pic/5}}
	\end{figure}
	\newpage
	\begin{figure}[h!]
		\center{\includegraphics[width=15cm]{pic/6}}
	\end{figure}
	\newpage
	\begin{figure}[h!]
		\center{\includegraphics[width=15cm]{pic/7}}
	\end{figure}
	\newpage
	\begin{figure}[h!]
		\center{\includegraphics[width=15cm]{pic/8}}
	\end{figure}
	\newpage
	\begin{figure}[h!]
		\center{\includegraphics[width=7cm]{pic/9}}
	\end{figure}
	\newpage
\section{Transmitter:}
	\begin{figure}[h!]
		\center{\includegraphics[width=15cm]{pic/10}}
	\end{figure}
	\newpage
	\begin{figure}[h!]
		\center{\includegraphics[width=15cm]{pic/11}}
	\end{figure}
	\newpage
	\begin{figure}[h!]
		\center{\includegraphics[width=15cm]{pic/12}}
	\end{figure}
	\newpage
	\begin{figure}[h!]
		\center{\includegraphics[width=15cm]{pic/13}}
	\end{figure}
	\newpage
	\begin{figure}[h!]
		\center{\includegraphics[width=15cm]{pic/14}}
	\end{figure}
	\newpage
	\begin{figure}[h!]
		\center{\includegraphics[width=15cm]{pic/15}}
	\end{figure}
	\newpage
	\begin{figure}[h!]
		\center{\includegraphics[width=15cm]{pic/16}}
	\end{figure}
	\newpage
	\begin{figure}[h!]
		\center{\includegraphics[width=15cm]{pic/17}}
	\end{figure}
	\newpage
	\begin{figure}[h!]
		\center{\includegraphics[width=15cm]{pic/18}}
	\end{figure}
	\newpage
	\begin{figure}[h!]
		\center{\includegraphics[width=15cm]{pic/19}}
	\end{figure}
	\newpage
	
\section{decoder.txt}
DECODE('A','M', 12)\\

DECODE('B','I', 2111)\\

DECODE('C','C', 2121)\\

DECODE('D','R', 211)\\

DECODE('E','O', 1)\\

DECODE('F','N', 1121)\\

DECODE('G','T', 221)\\

DECODE('H','L', 1111)\\

DECODE('I','E', 11)\\

DECODE('J','A', 1222)\\

DECODE('K','B', 212)\\

DECODE('L','D', 1211)\\

DECODE('M','F', 22)\\

DECODE('N','G', 21)\\

DECODE('O','H', 222)\\

DECODE('P','J', 1221)\\

DECODE('Q','K', 2212)\\

DECODE('R','P', 121)\\

DECODE('S','Q', 111)\\

DECODE('T','S', 2)\\

DECODE('U','U', 112)\\

DECODE('V','V', 1112)\\

DECODE('W','W', 122)\\

DECODE('X','X', 2112)\\

DECODE('Y','Y', 2122)\\

DECODE('Z','Z', 2211)\\

%*********	
	
	\begin{figure}[h!]
		\center{\includegraphics[width=15cm]{pic/4}}
	\end{figure}
\newpage
\end{document}