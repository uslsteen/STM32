\documentclass{article}
\usepackage[utf8]{inputenc}
\usepackage[14pt]{extsizes}
\usepackage[russian]{babel}
\usepackage{fullpage}
\usepackage{graphicx}
\graphicspath{{pictures/}}
\DeclareGraphicsExtensions{.pdf,.png,.jpg}
\usepackage[usenames]{color}
\usepackage{colortbl}
\usepackage[hidelinks]{hyperref,xcolor}
\usepackage{setspace,amsmath}
\usepackage{indentfirst}

\begin{document}
	\begin{titlepage}
			\begin{center}
				\large 	Московский физико-технический институт \\
				Физтех-школа радиотехники и компьютерных технологий \\
				\vspace{0.2cm}
				
				\vspace{4.5cm}
				\large (Микроконтроллеры) \\ \vspace{0.2cm}
				\LARGE \textbf{100 вопросов}
			\end{center}
			\vspace{2.3cm} \large
			
			\begin{center}
				Работу выполнила: \\
				Руденко Варвара,
				Б01-901
				\vspace{10mm}		
				
			\end{center}
			
			\begin{center} \vspace{50mm}
				г. Долгопрудный \\
				2021 год
			\end{center}
		\end{titlepage}
		
\paragraph{Ответы на "100 вопросов":}
	\begin{enumerate}
		\item Руденко Варвара Дмитриевна, Б01-901
		\item Донов Геннадий Иннокентьевич
		\item Отличием микропроцессора от микроконтроллера является то, что микропроцессор - чистое ядро. У него отсутствуют: таймеры, порты ввода - вывода, аналого-цифровой преобразователь.
		\item У ATmega8535 могут быть тактовыми частотами:\\ от внешнего генератора (кристалл, RC-цепь) - 0,1 - 16 МГц; \\ от внутреннего генератора - 1, 2 и 4 МГц
		\item У ATmega8535 могут быть тактовыми таймерами: 2 восьмиразрядных и один 16-разрядный таймер
		\item Внутренняя структура МК:современные МК имеют структуру показанную на рисунке, блоки выполняют следующие функции:\\
		Процессор: обработка информации в соответствиии с системой команд МК\\
		Память программ: хранит программу, в соответствии с которой работает МК\\
		ОЗУ (RAM - Random Access Memory): хранение промежуточных результатов\\
		Порты ввода/вывода: обменивается информацией с внешним миром\\
		Блок управления питанием: обеспечивает правильный запуск после включения питания\\
		Блок управления сбросом: с входом RESET устанавливает МК в исходное состояние\\
		Блок синхронизации: вырабатывает тактовые сигналы, для правильного взаимодействия всех внутренних блоков
		\begin{figure}[h!]
			\center{\includegraphics[width=15cm]{pic/6}}
		\end{figure}
\newpage
		\item После сигнала RESET в TCCR будут записаны все нули
		\item Для порта А - восьмиразрядный двунаправленный порт ввода/вывода. Каждый разряд порта может быть настроен индивидуально в качестве либо входа, либо выхода. Если вывод = выход, то нагрузка тока до 20 мА. Выводы = входы аналого-цифрового преобразователя.
		\item Регистр SREG - 8-разрядный регистр (регистр флагов). Назначение приведено на рисунке:
		\begin{figure}[h!]
			\center{\includegraphics[width=15cm]{pic/5}}
		\end{figure}
\newpage
		\item Все прерывания после сигнала RESET запрещены, чтобы корректно инициализировать работу МК
		\item Разряд T в регистре SREG: BST r16, 4 (4-ый бит r16 -> 1-ый бит r17)
		\item Таймер 0:\\ Режимы работы:
			\begin{enumerate}
				\item Normal (режим 0) - TCNT0 суммирующий счетчик
				\item Phase Correct PWM (режим 1) – режим ШИМ с точной фазой
				\item  CTC (режим 2) – Clear Timer on Compare Match, режим счета по модулю, который определяется содержимым регистра OCR0 
				\item Fast PWM (режим 3) – быстродействующий ШИМ, позволяет генерировать высокочастотный сигнал
			\end{enumerate}
			Регистры ввода/вывода:\\
			Имеет 3 регистра ввода/вывода и еще 2 использует совместно с таймерами 1 и 2. Может использовать 2 вывода микроконтроллера: вход Т0 (Timer/Counter0 External Counter Input – вывод PBO), выход ОС0 (Timer/Counter0 Output Compare Match Output – вывод РВ3). 
			\begin{enumerate}
				\item Регистр контроля TCCR0 (Timer/Counter Control Register)
				\item обнуления SFIOR (Special Function IO Register), адрес \$30 в пространстве адресов ввода/вывода.
				\item Регистр прерывания, TIMSK (Timer/Counter Interrupt Musk Register), по переполнению в таймере 0.
				\item TIFR (Timer/Counter Interrupt Flag Register) - Регистр флагов прерываний таймеров/счетчиков  с адресом \$38 в пространстве адресов ввода/вывода.
			\end{enumerate}
			Количество прерываний: \\
2: по  сравнению и переполнению. 
			\begin{figure}[h!]
				\center{\includegraphics[width=15cm]{pic/2}}
			\end{figure}
\newpage
		\item Phase Correct PWM (режим 1) – режим ШИМ с точной фазой  
Режим Fast PWM (режим 3) – быстродействующий ШИМ, позволяющий генерировать высокочастотный сигнал с широтно-импульсной модуляцией. 
		\item На вход Т0 могут подаваться внешние тактовые импульсы (если предусмотрено, вывод РВ0 записать как вход, записав 0 в соответствующий разряд регистра DDRB), так же сигналы на вход приходят с управляемого предварительного делителя частоты (prescaler). 
		\item out TIMSK, r16 \\
ldi r16, 1(0) << TOIE0 
		\item \begin{figure}[h!]
			\center{\includegraphics[width=15cm]{pic/7}}
		\end{figure}
\newpage
		\item 1, 8, 64, 256, 1024 
		\item в Normal (режим 0) и CTC (режим 2) - поставить OC0 изменяться при совпадении с порогом.  
в ШИМ режимах (Phase Correct PWM(1) и Fast PWM(3)) - выставить порог в половину максимального (скважность = 0.5) 
		\item     написать в биты 2:0 регистра TCCR0 значение от 1 до 5
		\begin{figure}[h!]
			\center{\includegraphics[width=15cm]{pic/3}}
		\end{figure}
\newpage
		\item Normal (режим 0) - TCNT0 суммирующий счетчик, по каждому импульсу тактового сигнала значение увеличивается на 1. При переходе через \$FF возникает прерывание по переполнению, и счет продолжается со значения \$00. При совпадении содержимого счетчика TCNT0 и регистра порога OCR0, устанавливается в «1» флаг OCF0 и прерывание (если разрешено) начинает обрабатываться.  
Счетчик считает от 0 до 255. Генерируются прерывания по переполнении и по сравнении. 
		\item Phase Correct PWM (режим 1) – режим ШИМ с точной фазой, предназначен для генерации сигналов с широтно-импульсной модуляцией. TCNT0 функционирует как реверсивный счетчик, изменение состояния которого осуществляется по каждому импульсу тактового сигнала, поступающего от предварительного делителя. После достижения счетчиком мин/макс значения осуществляется прерывание и смена направления. После достижения \$00 устанавливается в «1» флаг прерывания TOV0 регистра TIFR. При совпадении содержимого счетчика TCNT0 и регистра порога OCR0, устанавливается в «1» флаг OCF0 и изменяется состояние выхода OC0. Особенность – двойная буферизация записи в регистр OCR0. Записываемое число сохраняется в специальном буферном регистре. Изменение содержимого регистра порога происходит после достижения TCNT0 макс значения \$FF. 
Счетчик считает от 0 до 255 и обратно. Генерируется прерывание по переполнении при проходе 0 
		\item CTC (режим 2) – Clear Timer on Compare Match, режим счета по модулю, который определяется содержимым регистра OCR0. TCNT0 обнуляется после того как его содержимое сравняется с содержимым регистра OCR0. Далее счет продолжается от \$00 до нового совпадения с порогом. При совпадении содержимого счетчика TCNT0 и регистра порога OCR0, устанавливается в «1» флаг OCF0 и прерывание (если разрешено) начинает обрабатываться. \\
Счетчик считает от 0 до OCR0. Генерируется прерывание по сравнении и при OCR0=255 полностью совпадающим с режимом 0 
		\item Fast PWM (режим 3) – быстродействующий ШИМ, позволяет генерировать высокочастотный сигнал. (использование: регулирование мощности, выпрямление, цифроаналоговое преобразование). Особенность – двойная буферизация записи в регистр OCR0. Записываемое число сохраняется в специальном буферном регистре. Изменение содержимого регистра порога происходит после достижения TCNT0 макс значения \$FF. 
Счётчик считает, как в режиме 0. Прерывание по сравнении генерируется один раз за период (можно сконфигурировать OC0, чтобы обнулялся при переполнении и сбрасывался при сравнении: получится желаемый ШИМ) 
		\item Состояние счетчика TCNT0 изменяется от \$00 до \$FF, после чего он обнуляется и счет повторяется. При переходе к состоянию \$00 устанавливается флаг прерывания TOV0 в регистре TIFR. 
		\item Запись и чтение в TCNT0 можно осуществлять без остановки счета
		\item После сигнала RESET регистр TCCR0 содержит все 0, таймер остановлен счета нет.  Запись всех 0 в младшие разряды ведет к остановке таймера. (CS02=0, CS01=0, CS00=0) 
		\item При выполнении некой программы иногда возникают события или запросы прерывания (нажатие кнопки INT0 или переполнение таймера). В результате чего система прерываний должна остановить работу основной программы и запустить программу обработки прерываний. Для каждого действия – своя.   
Все запросы поступают на блок обработки (Interrupt Unit), который определяет номер запроса (1-21) и возможность выполнения. В случае разрешенного прерывания чувствительность ко всем прерываниям запрещается (в 7-й разряд регистра флагов записывается 0). Текущее содержимое записывается в стек, на его место заносится адрес прерывания из таблицы векторов прерываний. 
Если необходимо несколько прерываний, то они будут выполнены в порядке приоритета от наименьшего номера. \\
RJMP - команда к началу прерывания (относится к командам безусловной передачи управления) \\
NOP- нет операции \\
RETI – возврат из прерывания (относится к командам безусловной передачи управления).	
		\begin{figure}[h!]
			\center{\includegraphics[width=15cm]{pic/4}}
		\end{figure}
\newpage
		\item 21 прерывание. 4 – внешние вызываются сигналами, приходящими на выводы микроконтроллера INT0, INT1, INT2, RESET. 17-внутренние, обслуживают дополнительные блоки.
		\item Разрешение прерываний можно дать в начале программы обработки прерывания, тогда возможны вложенные прерывания (возможно переполнение памяти, ограничитель 512байт).
		\item Прерывания не будут обрабатываться, если в разряде 7 регистра флагов стоит 0, т.е общее запрещение прерываний. Факт возникновения соответствующего сигнала на входе запоминается в регистре GIFR (General Interrupt Flag Register). Разряд 7 этого регистра – флаг INT1, разряд 6 – INT0, разряд 5 – INT2.   
sei(cli) - разрешить(запретить) прерывание глобально  (SEI – (Global Interrupt Enable) устанавливает в 1 флаг глобального разрешения прерываний (разряд 7 регистра SPEG). Команда CLI (Global Interrupt Enable) устанавливает разряд 7 регистра SPEG в 0. 
		\item Первым будет обрабатываться прерывание имеющее наименьший номер в таблице векторов прерываний.
		\item Аналого-цифровой преобразователь, входит в состав микроконтроллера, обеспечивает преобразование аналогового напряжения в 10-разрядный двоичный код методом поразрядного уравновешивания. Мин время преобразования 65 микросек. 
		\item Сигнальный процессор обеспечивает обработку информации, выполняя команды в соответствии с системой команд микроконтроллера. 
МК – интегральная схема, которая может принимать сигналы от датчиков, обрабатывать и выдавать управляющие сигналы на исполнительные механизмы для выполнения поставленной задачи (работает с периферией). 
		\item Указатель стека SP (Stack Pointer) предназначен для работы со стеком, имеет 10 разрядов, состоит из 2-х 8-разрядных регистров (SPH-старший байт, SPL-младший байт). Обращение через команды IN, OUT. После команды RESET значение 0. 
Текущее содержимое SP определяет положение вершины стека. 
		\item WatchDog Timer – предназначен для ликвидации сбоев в работе МК, возникающих из-за различного рода помех. WDT через определенный заданный промежуток времени вырабатывает сигнал сброса (RESET) МК, перезапуская рабочую программу. Для обнаружения сбоев и предотвращения перезапуска при правильной работе в нее включают команду WDR (Watch Dog Reset) осуществляющей сброс сторожевого таймера, в результате отсчет времени начинается заново. 
		\item Последовательный синхронный интерфейс SPI - serial peripheral interface или интерфейс связи устройств, позволяет передавать данные с высокой скоростью между МК и внешними устройствами. Свойства:
		\begin{enumerate}
			\item Полнодуплексная (одновременно в 2-х направлениях) 3-х проводная синхронная передача данных. 
			\item Предельная скорость передачи данных СК/4 бит/сек 
			\item Передача осуществляется байтами.
			\item Передавать можно старшим либо младшим битом вперед  
			\item По окончании вырабатывается прерывание (адрес \$008) 
			\item Имеется флаг конфликтов при записи WCOL (Write Collision Flag) 
		\end{enumerate}		  
		\item Для нормального подключения необходимо:  \\
Для MASTER настроить MOSI, SCL, SS на выход, MISO на вход. \\  
Для SLAVE настроить MOSI, SCL, SS на выход, MISO на выход. \\  
При соединении одноименные выводы подключаются друг к другу \\
Выставив ~SS на ведомом устройстве в 0 
\begin{figure}[h!]
	\center{\includegraphics[width=15cm]{pic/1}}
\end{figure}
		\item Одно прерывание: SPIE – Interrupt Enable. Разрешение прерывания после передачи байта. (SPE – SPI Enable. Разрешение работы SPI. Если в этом разряде 0, то никакие ф-ции SPI не будут реализованы) \\ 
		3 регистра: 
		\begin{enumerate}
			\item SPI STATUS REGISTER (SPSR) - контрольный, можно использовать только для чтения, после RESET все 0 
			\item SPI CONTROL REGISTER (SPCR) - состояния - можно использовать для чтения и записи, после RESET все 0 
			\item SPI Data Register (SPDR) - под данные - можно использовать для чтения и записи, после RESET все 0 
		\end{enumerate} 
		\item -
		\item Для уменьшения количества физических соединений в микропроцессорных системах энергонезависимая память, устройства контроля доступа, датчики температуры, цифровые переключатели, мониторы аккумуляторных батарей и многие другие узлы часто подключаются с помощью всего двух проводов, используемых как для питания, так и передачи информации. Поскольку один из проводов является общим, то такой способ подключения стал называться однопроводным.
		\item Физический ноль - низкое напряжение и физическая единица - высокое
		\item В однопроводном интерфейсе передаются информационный ноль и информационная единица - логически; максимальная скорость  передачи 0 - длинный импульс физического нуля (60 мкс), 1 - короткий (15 мкс)
		\item Серийный номер в однопроводном интерфейсе - 64 бита: 8 бит - код семейства, 48 бит - серийный номер, 8 бит - контрольная сумма - уникальный идентификатор устройства, чтобы можно было выбрать устройство.
		\item Команда позволяющая Master определить номера всех Slave в сети MicroLAN - Search ROM
		\item Сигнал сброса в сети MicroLAN: Долгий импульс нуля (480 мкс), потом долгий импульс единицы, в течении которой master проверяет, есть ли кто-нибудь в сети.
	\end{enumerate}
\end{document}