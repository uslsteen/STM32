\subsection{Внутренняя структура МК.}
\subsection{Какие значения записаны в TCCR после сигнала RESET.}
\subsection{Порт А. Сколько прерываний и сколько регистров ввода/вывода принадлежит порту А. Назначение этих регистров ввода/вывода.}
\subsection{Регистр SREG. Назначение его  разрядов.}
\subsection{Почему после сигнала RESET все прерывания запрещены.}
\subsection{Приведите пример использования разряда Т в регистре SREG.}
\subsection{Таймер 0. Режимы работы, количество прерываний, регистры ввода/вывода, принадлежащие таймеру 0.}
\subsection{В каких режимах таймера 0 порог изменяется не сразу (двойная буферизация записи) при записи нового значения в регистр порога с помощью команды OUT. }
\subsection{Откуда приходит сигнал на вход TCNTO.}
\subsection{Как можно разрешить (запретить) прерывания по переполнению таймера 0.}
\subsection{Написать программу с использованием таймера 0, вырабатывающую симметричное прямоугольное колебание на одном из выходов порта А.}
\subsection{Какие коэффициенты деления частоты позволяет получать предварительный делитель таймера 0.}
\subsection{Какой режим таймера 0 позволяет вырабатывать треугольные колебания, используя дополнительную интегрирующую цепочку.}
\subsection{Как запрограммировать предварительный делитель таймера 0.}
\subsection{Режим 0 таймера 0.}
\subsection{Режим 1 таймера 0.}
\subsection{Режим 2 таймера 0.}
\subsection{Режим 3 таймера 0.}
\subsection{Когда меняется порог в режиме 3 таймера 0.}
\subsection{Можно ли писать в TCNT0 без остановки счета.}
\subsection{Как можно остановить счет в таймере 0.}
\subsection{Система прерываний микроконтроллера ATmega8535.}
\subsection{Сколько всего прерываний у ATmega8535.}
\subsection{Как организовать вложенные прерывания.}
\subsection{Как можно разрешить (запретить) одновременно все прерывания.}
\subsection{Как организована система приоритетов при обработке прерываний. }
\subsection{Какое минимальное время требуется для преобразования в АЦП.}
\subsection{Чем сигнальный процессор отличается от МК.}
\subsection{Зачем в программе надо устанавливать начальное значение Stack Pointer и чему это значение должно быть равно.}
\subsection{Сторожевой таймер и особенности его работы.}
\subsection{Что такое SPI и зачем он нужен.}
\subsection{Как инициировать передачу байта в SPI.}
\subsection{Сколько прерываний и сколько регистров ввода/вывода принадлежит SPI.}
\subsection{Далее пойдут вопросы про однопроводный интерфейс (сеть MicroLAN).}
\subsection{Сколько проводов необходимо для реализации однопроводного интерфейса.}
\subsection{Как выглядит физический ноль и физическая единица.}
\subsection{Как в однопроводном интерфейсе передается информационный ноль и информационная единица? Какова максимальная скорость  передачи?}
\subsection{Что такое серийный номер в однопроводном интерфейсе и какова его структура.}
\subsection{Какая команда позволяет Master определить номера всех Slave в сети MicroLAN.}
\subsection{Как выглядит сигнал сброса в сети MicroLAN.}